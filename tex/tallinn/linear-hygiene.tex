\documentclass[fleqn,aspectratio=169]{beamer}
\usetheme{Rochester}

\usepackage{cmll}
\usepackage{mathpartir}
\usepackage{mathtools}
\usepackage{stmaryrd}
\usepackage[style=alphabetic,citestyle=authoryear]{biblatex}
\addbibresource{../quantitative.bib}

\definecolor{use}{HTML}{008000}
\newcommand\gr[1]{{\color{use}#1}}
\newcommand\grctx[1]{\gr{\mathcal{#1}}}
\newcommand\grP{\grctx P}
\newcommand\grQ{\grctx Q}
\newcommand\grR{\grctx R}
\newcommand\grctxsub[2]{\grctx{#1}_{\gr{#2}}}
\newcommand\grPe{\grctxsub P e}
\newcommand\grPf{\grctxsub P f}
\newcommand\grQe{\grctxsub Q e}
\newcommand\grQf{\grctxsub Q f}
\newcommand\sem[1]{\left\llbracket{#1}\right\rrbracket}
\newcommand\ps{\mathit{ps}}
\newcommand\qs{\mathit{qs}}
\newcommand\rs{\mathit{rs}}
\newcommand\wand{\mathrel{\mathord{-}\hspace{-0.75ex}*}}
\newcommand\env[1]{(#1\mathrm{-Env})}
\newcommand\thinningN{\mathrm{Thinning}}
\newcommand\thinning[2]{\thinningN~#1~#2}
\newcommand\V{\mathcal V}
\newcommand\C{\mathcal C}
\newcommand\sqin{\mathrel{\mathrlap{\sqsubset}{\mathord{-}}}}
\newcommand\sqni{\mathrel{\mathrlap{\sqsupset}{\mathord{-}}}}
\newcommand\Ctx{\mathrm{Ctx}}
\newcommand\Ty{\mathrm{Ty}}
\newcommand\Set{\mathrm{Set}}

%\newenvironment{thinarray}%
%{\hskip -\arraycolsep\begin{array}[b]{c}}{\hskip -\arraycolsep\end{array}}

% Beamer hacks

\makeatletter

% Detect mode. mathpalette is used to detect the used math style
\newcommand<>\Alt[2]{%
  \begingroup
  \ifmmode
  \expandafter\mathpalette
  \expandafter\math@Alt
  \else
  \expandafter\make@Alt
  \fi
  {{#1}{#2}{#3}}%
  \endgroup
}

% Un-brace the second argument (required because \mathpalette reads the three arguments as one
\newcommand\math@Alt[2]{\math@@Alt{#1}#2}

% Set the two arguments in boxes. The math style is given by #1. \m@th sets \mathsurround to 0.
\newcommand\math@@Alt[3]{%
  \setbox\z@ \hbox{$\m@th #1{#2}$}%
  \setbox\@ne\hbox{$\m@th #1{#3}$}%
  \@Alt
}

% Un-brace the argument
\newcommand\make@Alt[1]{\make@@Alt#1}

% Set the two arguments into normal boxes
\newcommand\make@@Alt[2]{%
  \sbox\z@ {#1}%
  \sbox\@ne{#2}%
  \@Alt
}

% Place one of the two boxes using \rlap and place a \phantom box with the maximum of the two boxes
\newcommand\@Alt[1]{%
  \alt#1%
  {\rlap{\usebox0}}%
  {\rlap{\usebox1}}%
  \setbox\tw@\null
  \ht\tw@\ifnum\ht\z@>\ht\@ne\ht\z@\else\ht\@ne\fi
  \dp\tw@\ifnum\dp\z@>\dp\@ne\dp\z@\else\dp\@ne\fi
  \wd\tw@\ifnum\wd\z@>\wd\@ne\wd\z@\else\wd\@ne\fi
  \box\tw@
}

\makeatother

\title{A Hygiene and Framework for Linear Calculi with Binding}
\author{James Wood\inst{1} \and Bob Atkey\inst{1}}
\institute{\inst{1}University of Strathclyde}
\date{Tallinn, October 2020}

\begin{document}
\frame{\titlepage}
\begin{frame}
  \frametitle{Plan}
  \begin{enumerate}
    \item Solve syntax (up to and including substitution)
    \item Leave interesting stuff (semantics\footfullcite{AACMM20}) to the user
  \end{enumerate}

  Which syntaxes?
  \begin{itemize}
    \item Variables annotated by elements of a (skew) semiring describing
      \emph{usage}
    \item $\gr rx : A, \ldots, \gr sy : B \vdash \mathcal J$
    \item $(0, +)$ --- weakening/contraction, delete/duplicate
    \item $(1, *)$ --- use/modality, change of context
    \item $\trianglelefteq$ --- general/specific, if
      $\gr r \trianglelefteq \gr s$ and we have $\gr rx : A$, we can have
      $\gr sx : A$.
  \end{itemize}
\end{frame}
\begin{frame}
  \frametitle{A syntax for syntaxes}
  \only<1>{
    \begin{mathpar}
      \inferrule*[sep=1em]
      {\ps}
      {\mathcal J}
    \end{mathpar}

    \begin{mathpar}
      \inferrule*
      {\grR\Gamma \vdash \ps}
      {\grR\Gamma \vdash \mathcal J}
    \end{mathpar}
  }
  \only<2>{
    \begin{mathpar}
      \inferrule*[sep=1em]
      {\ps \\ * \\ \qs}
      {\mathcal J}
    \end{mathpar}

    \begin{mathpar}
      \inferrule*
      {
        \grP\Gamma \vdash \ps
        \\ \grR \trianglelefteq \grP + \grQ \\
        \grQ\Gamma \vdash \qs
      }
      {\grR\Gamma \vdash \mathcal J}
    \end{mathpar}
  }
  \only<3>{
    \begin{mathpar}
      \inferrule*[sep=1em]
      {\ps \\ * \\ (\qs \\ \wedge \\ \rs)}
      {\mathcal J}
    \end{mathpar}

    \begin{mathpar}
      \inferrule*
      {
        \grP\Gamma \vdash \ps
        \\ \grR \trianglelefteq \grP + \grQ \\
        {\hskip -\arraycolsep
          \begin{array}[b]{c}
            \grQ\Gamma \vdash \qs \\
            \grQ\Gamma \vdash \rs
          \end{array}
          \hskip -\arraycolsep}
      }
      {\grR\Gamma \vdash \mathcal J}
    \end{mathpar}
  }
  \only<4>{
    \begin{mathpar}
      \inferrule*[sep=1em]
      {A \oplus B \\ * \\ (\qs \\ \wedge \\ \rs)}
      {C}
    \end{mathpar}

    \begin{mathpar}
      \inferrule*
      {
        \grP\Gamma \vdash A \oplus B
        \\ \grR \trianglelefteq \grP + \grQ \\
        {\hskip -\arraycolsep
          \begin{array}[b]{c}
            \grQ\Gamma \vdash \qs \\
            \grQ\Gamma \vdash \rs
          \end{array}
          \hskip -\arraycolsep}
     }
     {\grR\Gamma \vdash C}
    \end{mathpar}
  }
  \only<5>{
    \begin{mathpar}
      \inferrule*[sep=1em]
      {
        {\hskip -\arraycolsep
          \begin{array}{c}\\\vphantom\vdots\\A \oplus B\end{array}
          \hskip -\arraycolsep}
        \\ {\hskip -\arraycolsep
          \begin{array}{c}\\\vphantom\vdots\\ *\end{array}
          \hskip -\arraycolsep} \\
        \left({\hskip -\arraycolsep
            \begin{array}[c]{ccc}
              \text{\footnotesize\([\gr1A]\)}
                     & & \text{\footnotesize\([\gr1B]\)} \\
              \vdots & & \vdots \\
              C & \wedge & C
            \end{array}
            \hskip -\arraycolsep}\right)
      }
      {C}
    \end{mathpar}

    \begin{mathpar}
      \inferrule*
      {
        \grP\Gamma \vdash A \oplus B
        \\ \grR \trianglelefteq \grP + \grQ \\
        {\hskip -\arraycolsep
          \begin{array}[b]{c}
            \grQ\Gamma, \gr1A \vdash C \\
            \grQ\Gamma, \gr1B \vdash C
          \end{array}
          \hskip -\arraycolsep}
      }
      {\grR\Gamma \vdash C}
    \end{mathpar}
  }
\end{frame}
\begin{frame}
  \frametitle{Premise combinators\footfullcite{RPKV20}}
  \begin{align*}
    \sem{\hskip -\arraycolsep
    \begin{array}{c}
      \text{\footnotesize\([\grQ\Delta]\)} \\
      \vdots \\
      \mathcal J
    \end{array}\hskip -\arraycolsep}~\grR\Gamma
                          &\coloneqq \grR\Gamma, \grQ\Delta \vdash \mathcal J \\
    \sem{\top}~\grR\Gamma &\coloneqq \top \\
    \sem{\ps \wedge \qs}~\grR\Gamma
                          &\coloneqq \sem\ps~\grR\Gamma
                            \;\wedge\; \sem\qs~\grR\Gamma \\
    \sem{I}~\grR\Gamma    &\coloneqq \grR \trianglelefteq \gr0 \\
    \sem{\ps * \qs}~\grR\Gamma
                          &\coloneqq \exists\grP,\grQ.~
                            \grR \trianglelefteq \grP + \grQ
                            \;\wedge\; \sem\ps~\grP\Gamma
                            \;\wedge\; \sem\qs~\grQ\Gamma \\
    \sem{\gr r \cdot \ps}~\grR\Gamma
                          &\coloneqq \exists\grP.~
                            \grR \trianglelefteq \gr r \grP
                            \;\wedge\; \sem\ps~\grP\Gamma
  \end{align*}
\end{frame}
\begin{frame}
  \frametitle{Bang!}
  \begin{align*}
    &\mathcal J \Coloneqq A \\
    &\{\gr0 \triangleright \gr\omega \triangleleft \gr1\},~
      \gr\omega \trianglelefteq \gr\omega + \gr\omega,~
      \gr\omega \trianglelefteq \gr0
  \end{align*}
  \begin{mathpar}
    \inferrule*
    {\gr\omega \cdot (M : A)}
    {[M] : \oc_{\gr\omega}A}
    \and
    \inferrule*
    {\Gamma; {\cdot} \vdash M : A}
    {\Gamma; {\cdot} \vdash [M] : \oc A}
  \end{mathpar}

  \begin{mathpar}
    \inferrule*[sep=1em]
    {
      M : \oc_{\gr\omega}A
      \\ * \\
      \overset{[\gr\omega x : A]}{\overset\vdots{N : C}}
    }
    {\textrm{let }x = [M]\textrm{ in }N : C}
    \and
    \inferrule*
    {\Gamma; \Delta_M \vdash M : \oc A \\ \Gamma, x : A; \Delta_N \vdash N : C}
    {\Gamma; \Delta_M, \Delta_N \vdash \textrm{let }x = [M]\textrm{ in }N : C}
  \end{mathpar}
\end{frame}
\begin{frame}
  \frametitle{Justifying the ancients of $\mu\tilde\mu$}
  $\mathcal J \Coloneqq
  \mathrm{command} \mid A~\mathrm{term} \mid A~\mathrm{coterm}$

  \only<1>{
    \begin{mathpar}
      \inferrule*
      {
        \overset{[\gr1\alpha : A~\mathrm{coterm}]}
        {
          \overset{\vdots}
          {c~\mathrm{command}}
        }
      }
      {\mu\alpha.~c : A~\mathrm{term}}
      \and
      \inferrule*
      {c : (\grP\Gamma \vdash \gr1\alpha : A, \grQ\Delta)}
      {\grP\Gamma \vdash \mu\alpha.~c : A \mid \grQ\Delta}
    \end{mathpar}

    \begin{mathpar}
      \inferrule*
      {
        \overset{[\gr1x : A~\mathrm{term}]}
        {
          \overset{\vdots}
          {c~\mathrm{command}}
        }
      }
      {\tilde\mu x.~c : A~\mathrm{coterm}}
      \and
      \inferrule*
      {c : (\grP\Gamma, 1x : A \vdash \grQ\Delta)}
      {\grP\Gamma \mid \tilde\mu x.~c : A \vdash \grQ\Delta}
    \end{mathpar}
  }
  \only<2>{
    \begin{mathpar}
      \inferrule*
      {
        \overset{
          [\gr1\alpha : A~\mathrm{coterm},~\gr1\beta : B~\mathrm{coterm}]
        }
        {
          \overset{\vdots}
          {c~\mathrm{command}}
        }
      }
      {\mu(\alpha,\beta).~c : A \parr B~\mathrm{term}}
      \and
      \inferrule*
      {c : (\grP\Gamma \vdash \gr1\alpha : A, \gr1\beta : B, \grQ\Delta)}
      {\grP\Gamma \vdash \mu(\alpha,\beta).~c : A \parr B \mid \grQ\Delta}
    \end{mathpar}

    \begin{mathpar}
      \inferrule*[sep=1em]
      {
        e : A~\mathrm{coterm}
        \\ * \\
        f : B~\mathrm{coterm}
      }
      {(e,f) : A \parr B~\mathrm{coterm}}
      \and
      \inferrule*
      {
        \grPe\Gamma \mid e : A \vdash \grQe\Delta
        \\
        \grPf\Gamma \mid f : A \vdash \grQf\Delta
      }
      {(\grPe + \grPf)\Gamma \mid (e,f) : A \parr B \vdash
        (\grQe + \grQf)\Delta}
    \end{mathpar}
  }
\end{frame}
\begin{frame}
  \frametitle{The variable rule}
  \begin{mathpar}
    \inferrule*
    {(x : \mathcal J) \in \Gamma \\ \grP \trianglelefteq \langle x \rvert}
    {\grP\Gamma \vdash x : \mathcal J}
  \end{mathpar}

  We write $\mathcal J \sqin \grP\Gamma$.
\end{frame}
\begin{frame}
  \frametitle{Generic semantics (easy mode)}
  \begin{columns}
    \begin{column}{0.5\textwidth}
      Let $d$ be our collection of rules.

      \begin{align*}
        \mathit{var} &: \only<1>{\forall \grR\Gamma.~
                       \V~A~\grR\Gamma \to \C~A~\grR\Gamma}
                       \only<2->{\forall[~\V~A \to \C~A~]} \\
        \mathit{alg} &: \only<1>{\forall \grR\Gamma.~
                       \sem d\,\C~A~\grR\Gamma \to \C~A~\grR\Gamma}
                       \only<2->{\forall[~\sem d\,\C~A \to \C~A~]} \\
      \end{align*}

      We want to derive:
    \end{column}
    \begin{column}{0.5\textwidth}
      \only<3>{
        \begin{block}{Examples}
          ``Syntactic'': $\C~A~\grR\Gamma \coloneqq \grR\Gamma \vdash_d A$
          \begin{itemize}
            \item Renaming: $\V~A~\grR\Gamma \coloneqq A \sqin \grR\Gamma$
            \item Substitution:
              $\V~A~\grR\Gamma \coloneqq \grR\Gamma \vdash_d A$
          \end{itemize}

          ``Semantic'':
          \vspace{-1em}
          \begin{align*}
            \C~A~\grR\Gamma &\coloneqq \sem{\grR\Gamma} \to \sem A \\
            \V~A~\grR\Gamma &\coloneqq A \sqin \grR\Gamma
          \end{align*}
          \vspace{-2em}
        \end{block}
      }
    \end{column}
  \end{columns}

  \begin{align*}
    \mathit{semantics} &: \forall \grP\Gamma, \grQ\Delta, A.~
                         \env{\grP\Gamma}~\V~\grQ\Delta \to
                         \grP\Gamma \vdash_d A \to \C~A~\grQ\Delta
  \end{align*}

  Intuitionistically, $\env{\Gamma}~\V~\Delta = A \in \Gamma \to \V~A~\Delta$.
\end{frame}
\begin{frame}
  \frametitle{Environments, linearly}
  Intuitionistically, $\env{\Gamma}~\V~\Delta = A \in \Gamma \to \V~A~\Delta$.

  To make $\env{\grP\Gamma}~\V~\grQ\Delta$, we add:
  \begin{itemize}
    \item a matrix $\gr\Psi$
    \item such that $\grQ \trianglelefteq \grP\gr\Psi$.
  \end{itemize}
  The lookup function now has type
  $(x : A) \in \Gamma \to \V~A~\gr\Psi_x\Delta$. \pause

  Special environments:
  $\thinning{\grP\Gamma}{\grQ\Delta} \coloneqq
  \env{\grP\Gamma}~(\sqin)~\grQ\Delta$

  Example: $\thinning{(\gr2x : A, \gr1x' : A, \gr2y : B)}%
  {(\gr0w : D, \gr2y : B, \gr0z : C, \gr3x : A)}$
  \begin{itemize}
    \item $\gr\Psi \coloneqq
      \begin{pmatrix}
        \gr0&\gr0&\gr0&\gr1\\
        \gr0&\gr0&\gr0&\gr1\\
        \gr0&\gr1&\gr0&\gr0
      \end{pmatrix}
      \begin{matrix}
        {} \sqni A \\ {} \sqni A \\ {} \sqni B
      \end{matrix}$
  \end{itemize}
\end{frame}
\begin{frame}
  \frametitle{Thinnings, linearly}
  \[\thinning{\grP\Gamma}{\grQ\Delta} \coloneqq
  \env{\grP\Gamma}~(\sqin)~\grQ\Delta\]
  \begin{align*}
    (\square T)~\grP\Gamma &\coloneqq \forall[~\thinningN~\grP\Gamma \to T~] \\
    \mathrm{Thinnable}~T   &\coloneqq \forall[~T \to \square T~] \\
    (S \wand T)~\grP\Gamma &\coloneqq \forall\grR,\grQ.~
                             \grR \trianglelefteq \grP + \grQ \to
                             S~\grQ\Gamma \to T~\grR\Gamma \\
    \mathrm{Kripke}~\grR\Theta~A
                           &\coloneqq \square(\env{\grR\Theta}~\V \wand \C~A)
  \end{align*}
\end{frame}
\begin{frame}
  \frametitle{Generic semantics (hard mode)}
  \begin{align*}
    \mathrm{Kripke}~\grR\Theta &: \Ty \to \Ctx \to \Set \\
    \mathrm{Kripke}~\grR\Theta~A~\grQ\Delta
    &\coloneqq \square(\env{\grR\Theta}~\V \wand \C~A)~\grQ\Delta
  \end{align*}

  A semantics respecting variable binding needs:
  \begin{align*}
    \mathit{th}^\V &: \forall A.~\mathrm{Thinnable}~(\V~A) \\
    \mathit{var} &: \forall[~\V~A \to \C~A~] \\
    \mathit{alg} &: \forall[~\sem d\,\mathrm{Kripke}~A \to \C~A~] \\
    \\
    \sem d &: (\Ctx \to \Ty \to \Ctx \to \Set) \to \Ty \to \Ctx \to \Set
  \end{align*}
\end{frame}
\begin{frame}
  \frametitle{Conclusion}
  \begin{itemize}
    \item Generalise AACMM20 to linear and modal systems.
    \item Syntax --- bunched logic
    \item Semantics --- linear algebra
    \item Future work:
      \begin{itemize}
        \item First order definition of environments
        \item Generic usage checking
        \item Partial/relational algebra
        \item Fusion properties
      \end{itemize}
  \end{itemize}
\end{frame}
\end{document}
