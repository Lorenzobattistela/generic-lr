\documentclass[fleqn]{beamer}

\usepackage{booktabs}
\usepackage{subcaption}

\usepackage{stmaryrd}
\usepackage{mathpartir}
\usepackage{amssymb}
\usepackage{mathrsfs}
\usepackage{cmll}
\usepackage{xcolor}
\usepackage{makecell}
\usepackage{tikz-cd}
\usepackage{nccmath}
\usepackage{mathtools}
\usepackage{graphicx}
\usepackage[style=alphabetic,citestyle=authoryear]{biblatex}
\addbibresource{../quantitative.bib}

\usetikzlibrary{positioning}
% \usetikzlibrary{decorations.markings}
\usetikzlibrary{arrows}

\newsavebox{\pullback}
\sbox\pullback{%
  \begin{tikzpicture}%
    \draw (0,0) -- (1em,0em);%
    \draw (1em,0em) -- (1em,1em);%
  \end{tikzpicture}%
}

\def\newelims{1}
\def\multnotation{1}
\ifx\newelims\undefined
  \def\newelims{0}
\fi
\ifx\multnotation\undefined
  \def\multnotation{0}
\fi

\newcommand{\fixme}[1]{{\color{red}\underline{\em{#1}}}}
\newcommand{\fixmeBA}[1]{{\color{red}{[Bob:{\em #1}]}}}


\newcommand{\bind}[2]{{#2}\{{#1}\}}
\newcommand{\ctx}[2]{%
  \if\multnotation0%
  #1^{\resctx{#2}}%
  \else%
  \resctx{#2}#1%
  \fi
}
\newcommand{\ctxvar}[3]{%
  \if\multnotation0%
  #1 \stackrel{\rescomment{#3}}: #2%
  \else%
  \rescomment{#3}#1 : #2%
  \fi%
}
\definecolor{res}{HTML}{008000}
\definecolor{resperm}{HTML}{008000}
\newcommand{\rescomment}[1]{{\color{res}#1}}
\newcommand{\rescommentperm}[1]{{\color{resperm}#1}}
\newcommand{\resctx}[1]{\rescomment{\mathcal{#1}}}
\newcommand{\resctxperm}[1]{\rescommentperm{\mathcal{#1}}}
\newcommand{\resmat}[1]{\rescomment{#1}}


\newcommand{\ann}[2]{#1 : #2}
\newcommand{\emb}[1]{\underline{#1}}

\newcommand{\base}[0]{\iota}

\newcommand{\fun}[2]{#1 \multimap #2}
\newcommand{\lam}[2]{\lambda #1.~#2}
\newcommand{\app}[2]{#1~#2}

\newcommand{\excl}[2]{\oc_{\rescomment{#1}} #2}
\newcommand{\bang}[1]{\left[#1\right]}
\newcommand{\bm}[4]{\if\newelims0%
\operatorname{bm}_{#1}(#2, \bind{#3}{#4})%
\else%
\mathrm{let}~\bang{#3} = #2~\mathrm{in}~#4%
\fi}

\newcommand{\tensorOne}[0]{1}
\newcommand{\unit}[0]{(\mathbin{_\otimes})}
\newcommand{\del}[3]{\if\newelims0%
\operatorname{del}_{#1}(#2, #3)%
\else%
\mathrm{let}~\unit = #2~\mathrm{in}~#3%
\fi}

\newcommand{\tensor}[2]{#1 \otimes #2}
\newcommand{\ten}[2]{(#1 \mathbin{_{\otimes}} #2)}
\newcommand{\prm}[5]{\if\newelims0%
\operatorname{pm}_{#1}(#2, \bind{#3, #4}{#5})%
\else%
\mathrm{let}~\ten{#3}{#4} = #2~\mathrm{in}~#5%
\fi}

\newcommand{\withTOne}[0]{\top}
\newcommand{\eat}[0]{(\mathbin{_{\with}})}

\newcommand{\withT}[2]{#1 \with #2}
\newcommand{\wth}[2]{(#1 \mathbin{_\with} #2)}
\newcommand{\proj}[2]{\operatorname{proj}_{#1} #2}

\newcommand{\sumTZero}[0]{0}
\newcommand{\exf}[2]{\operatorname{ex-falso} #2}

\newcommand{\sumT}[2]{#1 \oplus #2}
\newcommand{\inj}[2]{\operatorname{inj}_{#1} #2}
\newcommand{\cse}[6]{\if\newelims0%
\operatorname{case}_{#1}(#2, \bind{#3}{#4}, \bind{#5}{#6})%
\else%
\mathrm{case}~#2~\mathrm{of}~\inj{L}{#3} \mapsto #4
                         ~;~ \inj{R}{#5} \mapsto #6%
\fi}

\newcommand{\listT}[1]{\operatorname{List} #1}
\newcommand{\nil}[0]{[]}
\newcommand{\cons}[2]{#1 :: #2}
\newcommand{\fold}[5]{\operatorname{fold} #1~#2~(#3,#4.~#5)}


\newcommand{\typed}[1]{\mathit{#1t}}
\newcommand{\resourced}[1]{\mathit{#1r}}


\newcommand{\lvar}{\mathrel{\mathrlap{\sqsupset}{\mathord{-}}}}
\newcommand{\sem}[1]{\left\llbracket #1 \right\rrbracket}

%\def\tobar{\mathrel{\mkern3mu  \vcenter{\hbox{$\scriptscriptstyle+$}}%
%                    \mkern-12mu{\to}}}
\newcommand\tobar{\mathrel{\ooalign{\hfil$\mapstochar\mkern5mu$\hfil\cr$\to$\cr}}}


\newcommand{\subres}{\trianglelefteq}
\newcommand{\subrctx}[2]{{#1} \subres {#2}}
\makeatletter
\newcommand{\subst}[2][]{\ext@arrow 0359\Rightarrowfill@{#1}{#2}}
\makeatother


\newenvironment{eqns}{\begin{array}{r@{\hspace{0.3em}}c@{\hspace{0.3em}}l}}{\end{array}}


\newcommand{\mat}[1]{\mathbf{#1}}
\newcommand{\vct}[1]{\mathbf{#1}}
\DeclarePairedDelimiter\basis{\langle}{\rvert}


\newcommand{\name}{\ensuremath{\lambda \resctxperm R}}


\DeclareMathOperator\kit{Kit}
\newcommand{\kitrel}{\mathbin{\blacklozenge}}


\DeclareMathOperator\id{id}
\DeclareMathOperator\inl{inl}
\DeclareMathOperator\inr{inr}


\newcommand{\zero}{\ensuremath{\rescomment 0}}
\newcommand{\linear}{\ensuremath{\rescomment 1}}
\newcommand{\unrestricted}{\ensuremath{\rescomment \omega}}

\newcommand{\unused}{\ensuremath{\rescomment{\mathit{unused}}}}
\newcommand{\true}{\ensuremath{\rescomment{\mathit{true}}}}
\newcommand{\valid}{\ensuremath{\rescomment{\mathit{valid}}}}

\newcommand\dotto{\mathrel{\dot\to}}
\newcommand\dottimes{\mathbin{\dot\times}}
\newcommand\wand{\mathrel{\mathord{-}\hspace{-0.75ex}*}}
\newcommand\sep{\mathbin{*}}

\newcommand{\aval}[1]{\rescomment{\mathrm{#1}}}

% Beamer hacks

\makeatletter

% Detect mode. mathpalette is used to detect the used math style
\newcommand<>\Alt[2]{%
  \begingroup
  \ifmmode
  \expandafter\mathpalette
  \expandafter\math@Alt
  \else
  \expandafter\make@Alt
  \fi
  {{#1}{#2}{#3}}%
  \endgroup
}

% Un-brace the second argument (required because \mathpalette reads the three arguments as one
\newcommand\math@Alt[2]{\math@@Alt{#1}#2}

% Set the two arguments in boxes. The math style is given by #1. \m@th sets \mathsurround to 0.
\newcommand\math@@Alt[3]{%
  \setbox\z@ \hbox{$\m@th #1{#2}$}%
  \setbox\@ne\hbox{$\m@th #1{#3}$}%
  \@Alt
}

% Un-brace the argument
\newcommand\make@Alt[1]{\make@@Alt#1}

% Set the two arguments into normal boxes
\newcommand\make@@Alt[2]{%
  \sbox\z@ {#1}%
  \sbox\@ne{#2}%
  \@Alt
}

% Place one of the two boxes using \rlap and place a \phantom box with the maximum of the two boxes
\newcommand\@Alt[1]{%
  \alt#1%
  {\rlap{\usebox0}}%
  {\rlap{\usebox1}}%
  \setbox\tw@\null
  \ht\tw@\ifnum\ht\z@>\ht\@ne\ht\z@\else\ht\@ne\fi
  \dp\tw@\ifnum\dp\z@>\dp\@ne\dp\z@\else\dp\@ne\fi
  \wd\tw@\ifnum\wd\z@>\wd\@ne\wd\z@\else\wd\@ne\fi
  \box\tw@
}

\newcommand\ctxvarh[3]{\ctxvar{#1}{#2}{\uncover<2>{#3}}}

\makeatother

\title{Towards a Universe of Linear Syntaxes with Binding}
\author{James Wood\inst{1} \and Bob Atkey\inst{1}}
\institute{\inst{1}University of Strathclyde}
\date{VEST, June 2020}

\begin{document}
\setlength{\abovedisplayskip}{0pt}
\frame{\titlepage}
\begin{frame}
  \frametitle{Context}
  \begin{center}
    \begin{tikzcd}[ampersand replacement=\&, sep=large]
      \& |[label=center:\clap{McB05$^{1}$}]|{\phantom M}
      \arrow[ld, "\textrm{linearity}"'] \arrow[rd, "\textrm{semantics}"] \& \\
      |[label=center:\clap{WA20$^{2}$}]|{\phantom M} \arrow[rd] \&
      \& |[label=center:\clap{AACMM18/20$^{3}$}]|{\phantom M} \arrow[ld] \\
      \& |[label=center:\clap{WIP}]|{\phantom M} \arrow[uu, phantom, "\usebox\pullback" rotate=135, very near start] \&
    \end{tikzcd}
  \end{center}
  \begin{itemize}
    \item Linearity independent of binding
    \item Only one traversal over the syntax
      % Hacks:
      \phantom{\footfullcite{rensub05}
      \footfullcite{linear-metatheory}\footfullcite{AACMM20}}
  \end{itemize}
\end{frame}
\begin{frame}
  \frametitle{Idea --- stability under weakening \footfullcite{AACMM20}}
  Two parts:
  \begin{enumerate}
    \item Consolidate all traversals over syntax (e.g, simultaneous renaming,
      simultaneous substitution, NbE, printing) into a single generic traversal
      (\emph{kits/semantics}).
    \item Build typing rules from small building blocks, so that they admit a
      generic semantics by construction.
  \end{enumerate}
\end{frame}
\begin{frame}
  \frametitle{Generic notion of semantics}
  \begin{itemize}
    \item In the intuitionistic case, we have the following fundamental lemma of
      semantics:
      \[
      \inferrule*
      {\mathrm{Semantics}~\mathcal V~\mathcal C
      \\ \overbrace{\Gamma \ni A \to \mathcal V~A~\Delta}^{\textrm{environment}}}
      {{\underbrace{\Gamma \vdash A \to \mathcal C~A~\Delta}
      _{\textrm{traversal}}}}
      \]
    \item $\mathrm{Semantics}~\mathcal V~\mathcal C$ contains:
      \begin{itemize}
        \item Proof that $\mathcal V$ is stable under weakening
        \item Ways to interpret term constructors semantically, generic in
          context:
          \begin{mathpar}
            \sem{\mathrm{var}}
            : \forall[~\mathcal V~A \dotto \mathcal C~A~]
            \and
            \sem{\mathrm{app}}
            : \forall[~\mathcal C~(A \to B) \dottimes \mathcal C~A
            \dotto \mathcal C~B~]
            \and
            \sem{\mathrm{lam}}
            : \forall[~\square(\mathcal V~A \dotto \mathcal C~B)
            \dotto \mathcal C~(A \to B)~]
            \and
            \dots
          \end{mathpar}
      \end{itemize}
  \end{itemize}
\end{frame}
\begin{frame}
  \frametitle{Generic notion of syntax}
  \begin{itemize}
    \item A type system can:
      \begin{enumerate}
        \item Offer a multitude of term formers. e.g, \TirName{App},
          \TirName{Lam}, \ldots
        \item For each term former, require 0 or more premises. $\dottimes$,
          $\dot 1$
        \item For each premise, maybe bind variables. $\square$, $\mathcal V$
      \end{enumerate}
    \item Variables are a special case.
    \item Example descriptions:
      \begin{itemize}
        \item \phantom{.}\TirName{App}: $(A \to B) \times A \Rightarrow B$
        \item \phantom{.}\TirName{Lam}: $(A \vdash B) \Rightarrow (A \to B)$
      \end{itemize}
  \end{itemize}
\end{frame}
\begin{frame}
  \frametitle{Example derivation}
  \begin{flalign*}
    & P = \tensor{(\fun{A}{B})}{A} \\\\
    & \inferrule*{
      \inferrule*{
        \inferrule*[rightskip=5em,vdots=5em]{ }
        {\ctxvarh{p}{P}{1} \vdash p : P}
        \\
        \inferrule*{
          \inferrule*{ }
          {\ctxvarh{p}{P}{0}, \ctxvarh{f}{\fun{A}{B}}{1}, \\\\ \ctxvarh{x}{A}{0}
            \vdash f : \fun{A}{B}}
          \\
          \inferrule*{ }
          {\ctxvarh{p}{P}{0}, \ctxvarh{f}{\fun{A}{B}}{0}, \\\\ \ctxvarh{x}{A}{1}
            \vdash x : A}
        }
        {\ctxvarh{p}{P}{0}, \ctxvarh{f}{\fun{A}{B}}{1}, \ctxvarh{x}{A}{1}
          \vdash \app{f}{x} : B}
      }
      {\ctxvarh{p}{P}{1} \vdash \prm{B}{p}{f}{x}{\app{f}{x}} : B}
    }
    {\vdash \lam{p}{\prm{B}{p}{f}{x}{\app{f}{x}}}
    : \fun{\underbrace{\tensor{(\fun{A}{B})}{A}}_{P}}{B}}
  \end{flalign*}
\end{frame}
\begin{frame}
  \frametitle{Vectors over semirings --- addition}
  Semiring operations (operating on annotations on individual variables) are
  lifted to vector operations (operating on contexts-worth of variables).
  \vspace{1em}
  \begin{columns}
    \begin{column}{0.5\textwidth}
      \begin{mathpar}
        \inferrule*
        {\ctx{\Gamma}{P} \vdash M : A \\ \ctx{\Gamma}{Q} \vdash N : B
          \\\\ \resctx R \subres \resctx P + \resctx Q}
        {\ctx{\Gamma}{R} \vdash \ten{M}{N} : \tensor{A}{B}} \\
        \inferrule*
        {\resctx R \subres 0}
        {\ctx{\Gamma}{R} \vdash \unit{} : \tensorOne{}}
      \end{mathpar}
    \end{column}
    \begin{column}{0.5\textwidth}
      identity, associativity, commutativity $\sim$ contexts are
      essentially multisets
    \end{column}
  \end{columns}
\end{frame}
\begin{frame}
  \frametitle{Vectors over semirings --- multiplication}
  \begin{columns}
    \begin{column}{0.5\textwidth}
      \begin{mathpar}
        \inferrule*
        {\Gamma \ni (x : A) \\ \resctx R \subres \langle x \rvert}
        {\ctx{\Gamma}{R} \vdash x : A}
        \\
        \inferrule*
        {\ctx{\Gamma}{P} \vdash M : A
          \\ \resctx R \subres \rescomment r \resctx P}
        {\ctx{\Gamma}{R} \vdash \bang M : \excl{r}{A}}
      \end{mathpar}
    \end{column}
    \begin{column}{0.5\textwidth}
      \begin{itemize}
        \item $\langle x \rvert$ --- basis vector. The variable $x$ can be used
          once, and every other variable can be discarded.
        \item `M' for ``Multiplication'', also for ``Modality''
      \end{itemize}
    \end{column}
  \end{columns}
  \vspace{1em}
\end{frame}
\begin{frame}
  \frametitle{Vectors over semirings}
  \begin{mathpar}
    \inferrule*
    {\resctx R \subres 0}
    {\ctx{\Gamma}{R} \vdash \unit{} : \tensorOne{}}
    \and
    \inferrule*
    {\ctx{\Gamma}{P} \vdash M : A \\ \ctx{\Gamma}{Q} \vdash N : B
      \\\\ \resctx R \subres \resctx P + \resctx Q}
    {\ctx{\Gamma}{R} \vdash \ten{M}{N} : \tensor{A}{B}} \\
    \and
    \inferrule*
    {\Gamma \ni (x : A) \\ \resctx R \subres \langle x \rvert}
    {\ctx{\Gamma}{R} \vdash x : A}
    \and
    \inferrule*
    {\ctx{\Gamma}{P} \vdash M : A
      \\ \resctx R \subres \rescomment r \resctx P}
    {\ctx{\Gamma}{R} \vdash \bang M : \excl{r}{A}}
  \end{mathpar}
  \begin{itemize}
    \item These four are the basic operations of linear algebra, three of them
      preserved by linear transformations.
    \item Notice: a variable which is syntactically absent may be given
      annotation $\rescomment 0$.
  \end{itemize}
\end{frame}
\begin{frame}
  \frametitle{Generic notion of linear\footnote{for a generic notion of
      linearity} semantics}
  \begin{itemize}
    \item Fundamental lemma of semantics:
      \[
      \inferrule*
      {\mathrm{Semantics}~\mathcal V~\mathcal C
      \\ \overbrace{(i : \Gamma \ni A) \to
      \mathcal V~A~(\rescomment\Psi_{i-}\Delta)} ^{\textrm{environment}}
      \\ \resctx Q \subres \resctx P \rescomment\Psi}
      {{\underbrace{\ctx{\Gamma}{P} \vdash A \to \mathcal C~A~\ctx{\Delta}{Q}}
      _{\textrm{traversal}}}}
      \]
    \item $\mathrm{Semantics}~\mathcal V~\mathcal C$ contains:
      \begin{itemize}
        \item Proof that $\mathcal V$ is stable under weakening \emph{by
          \rescomment 0-use variables}
        \item Ways to interpret term constructors semantically, generic in
          context:
          \begin{mathpar}
            \sem{\mathrm{var}}
            : \forall[~\mathcal V~A \dotto \mathcal C~A~]
            \and
            \sem{\mathrm{app}}
            : \forall[~\mathcal C~(\fun{A}{B}) \sep \mathcal C~A
            \dotto \mathcal C~B~]
            \and
            \sem{\mathrm{lam}}
            : \forall[~\square(\mathcal V~A \wand{} \mathcal C~B)
            \dotto \mathcal C~(\fun{A}{B})~]
            \and
            \dots
          \end{mathpar}
      \end{itemize}
  \end{itemize}
\end{frame}
\begin{frame}
  \frametitle{Generic notion of linear syntax}
  Multiple premises are handled by bunched implications. \footfullcite{RPKV20}
  \begin{itemize}
    \item \(I~\ctx{\Gamma}{R} := \resctx R \subres 0\)
    \item \((\mathcal A \sep \mathcal B)~\ctx{\Gamma}{R} :=
      \Sigma \resctx P, \resctx Q.~(\resctx R \subres \resctx P + \resctx Q)
      \times \mathcal A~\ctx{\Gamma}{P} \times \mathcal B~\ctx{\Gamma}{Q}\)
    \item \((\mathcal A \wand \mathcal B)~\ctx{\Gamma}{P} :=
      \Pi \resctx Q, \resctx R.~(\resctx R \subres \resctx P + \resctx Q)
      \to \mathcal A~\ctx{\Gamma}{Q} \to \mathcal B~\ctx{\Gamma}{R}\)
    \item \((\oc_{\rescomment r}\mathcal A)~\ctx{\Gamma}{R} :=
      \Sigma \resctx P.~(\resctx R \subres \rescomment r\resctx P)
      \times \mathcal A~\ctx{\Gamma}{P}\)
  \end{itemize}
  Example description:
  $(\excl{r}{A} \sep (\rescomment rA \vdash B)) \Rightarrow B$
  \begin{itemize}
    \item \(
      \inferrule
      {\ctx{\Gamma}{P} \vdash \excl{r}{A}
      \\ \ctx{\Gamma}{Q}, \ctxvar{x}{A}{r} \vdash B
      \\ \resctx R \subres \resctx P + \resctx Q}
      {\ctx{\Gamma}{R} \vdash B}
      \)
    \item $\sem{\mathrm{bam}} :
      \forall[~\mathcal C~(\excl{r}{A}) \sep
      \square(\excl{r}{(\mathcal V~A)} \wand \mathcal C~B)
      \dotto \mathcal C~B~]$
  \end{itemize}
\end{frame}
\begin{frame}
  \frametitle{Conclusion}
  \begin{itemize}
    \item Past:
      \begin{itemize}
        \item Intuitionistic syntax-to-syntax traversals using kits.
        \item Intuitionistic syntax-to-semantics traversals.
        \item A generic notion of intuitionistic syntax with binding.
      \end{itemize}
    \item Present:
      \begin{itemize}
        \item Linear syntax-to-syntax traversals.
        \item Using matrices/linear maps to describe compatibility of
          environments and traversals.
      \end{itemize}
    \item Future:
      \begin{itemize}
        \item Linear syntax-to-semantics traversals using matrices.
        \item Semantics is stable under $\rescomment 0$-use weakening.
        \item A generic notion of linear syntax written in bunched implications
          style.
        \item \url{https://github.com/laMudri/generic-lr}
      \end{itemize}
  \end{itemize}
\end{frame}

\end{document}
